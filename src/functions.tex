\section{Algebra}

\formdesc{Arithmetic Operations}
\vspace{-3.5em}
\begin{center}
\[
\def\arraystretch{2.5}
 \begin{array}{ll}
  a \left( \frac{b}{c} \right) = \frac{ab}{c}   & \frac{\frac{a}{b}}{c} = \frac{a}{bc}               \\
   \frac{a}{\frac{b}{c}} = \frac{ac}{b}         & \frac{a}{b} \pm \frac{c}{d} = \frac{ad \pm bc}{bd} \\
   \frac{a - b}{c - d} = \frac{b - a}{d - c}    & \frac{ab + ac}{a} = b + c,~ \forall a \neq 0        \\
   \frac{a + bc}{c} = \frac{a}{c} = \frac{b}{c} & \frac{\frac{a}{b}}{\frac{c}{d}} = \frac{ad}{bc} 
 \end{array}
\]
\end{center}
\hformbar



\formdesc{Exponents}

\vspace{-3.5em}
\begin{center}
\[
\def\arraystretch{2.5}
 \begin{array}{ll}
   x^a \cdot x^b = x^{a + b}           & \frac{x^a}{x^b} = x^{a - b}     \\
   (xy)^a = x^a y^a                    & \frac{x}{y}^a = \frac{x^a}{y^a} \\
   x^0 = 1 \note                       & x^{-n} = \frac{1}{x} \note      \\
   x^{\frac{1}{a}} = \sqrt[a]{x} \note & (xy)^a = x^a y^a                \\
   x^{-a} = \frac{1}{x^a}             & \left( \frac{x}{y}   \right)^{-a} = \left( \frac{y}{x} \right)^a = \frac{y^a}{x^a} 
 \end{array}
\]
\end{center}

$\note \forall x \neq 0$

\hformbar



\formdesc{Radicals}

\vspace{-2.5em}
\begin{center}
\[
\def\arraystretch{2.5}
 \begin{array}{ll}
   \sqrt[a]{x} = x^{\frac{1}{a}} & \sqrt[a]{xy} = \sqrt[a]{x} \sqrt[a]{y}
 \end{array}
\]
\end{center}

\hformbar



\formdesc{Logarithms}

Defined as $y = log_b x \equiv x = b^y$, with domain of $x > 0$

\begin{center}
\vspace{-2em}
\[
\def\arraystretch{1.6}
 \begin{array}{ll}
   log_b b = 1 & log_b 1 = 0 \\
   log_b b^x = x & b^{log_b x} = x \\
   log_b \left( x^r \right) = r ~ log_b x &  \\
   log_b \left( \frac{x}{y} \right) = log_b x - log_b y & \\
   log_b (xy) = log_b x + log_b y & 
 \end{array}
\]
\end{center}

\hformbar



\formdesc{Factoring}

\begin{center}
\vspace{-2em}
\[
\def\arraystretch{1.6}
 \begin{array}{c}
   x^2 - a^2 = (x + a)(x -a ) \\
   x^2 \pm 2ax + a^2 = (x \pm a)^2 \\
   x^2 + (a + b)x + ab = (x + a)(x + b)
 \end{array}
\]
\end{center}

\hformbar





\formdesc{Functions}

\begin{itemize}
	\item A rule for a relationship between an input and an output quantity where each input \textit{uniquely determines} an output value 
	\item Must be of the form $y = f(x)$
\end{itemize}

\begin{footnotesize}
  \begin{tabular}{ll}
    \textsc{type} & \textsc{form} \\
    \midrule
    Constant & $f(x) = c$ \\
    Identity & $f(x) = x$ \\
    Absolute & $f(x) = |x|$\\
    Quadratic & $f(x) = x^2$  \\
    Cubic     & $f(x) = x^3$  \\
    Reciprocal & $f(x) = \frac{1}{x^2}$  \\
    Square root & $f(x) = \sqrt{x}$  \\
  \end{tabular}
\end{footnotesize}
\hformbar



\formdesc{Transformations of Functions}

\vspace{.3em}
\begin{footnotesize}
  \begin{tabular}{ll}
  \centering
    \textsc{Transformation} & \textsc{Form} \\
    \midrule
    Vertical shift & $f'(x) = f(x) + k$ \\
    Horizontal shift & $f'(x) = f(x - k)$\\
    Horizontal reflection & $f'(x) = f(-x)$\\
    Vertical reflection & $f'(x) = -f(x)$  \\
    Vertical stretch     & $f'(x) = kf(x), ~ k > 1$ \\
    Vertical compression & $f'(x) = kf(x), ~ k 0 < k < 1$  \\
  \end{tabular}
\end{footnotesize}
\hformbar



\formdesc{Distance Formula}

\begin{equation}
	d(P_1, P_2) = \sqrt{(x_2 - x_1)^2 + (y_2 - y_1)^2}
\end{equation}


is the distance between points $P_1 = (x_1, y_1)$ and $P_2 = (x_2, y_2)$
\hformbar



\formdesc{Linear Functions}

\begin{itemize}
	\item Linear functions have a constant rate of change, $m$

		\begin{equation}
			m = \frac{\Delta y}{\Delta x} = \frac{y_2 - y_1}{x_2 - x_1} = \frac{f(x_2) - f(x_1)}{x_2 - x_1}
		\end{equation}
	\item Estimate horizontal line by solving $f(x) = 0$
	\item Find the point where two non-parallel lines meet: set $f(x) = g(x)$ and solve for $x$
	\item Estimate slope from two points by solving for $m$:
	    \begin{equation}
	        y_1 - y_2 = m(x_1 - x_2)
	    \end{equation}
\end{itemize}
\hformbar



\formdesc{Quadratic Functions}

\begin{center}
  \begin{tabular}{ll}
    Standard form                  & $f(x) = ax^2 + bx + c$                   \\[.7em]
    Vertex or transformation form  & $f(x) = a(x - h)^2 + k$                  \\[.7em]
    Quadratic eq.                  & $x = \frac{-b \pm \sqrt{b^2 - 4ac}}{2a}$ \\[.7em]
  \end{tabular}
\end{center}

\begin{itemize}
    \item $b^2 - 4ac > 0 \Rightarrow$ Two real unequal solutions
    \item $b^2 - 4ac = 0 \Rightarrow$ Repeated real solution
    \item $b^2 - 4ac < 0 \Rightarrow$ Two complex solutions
\end{itemize}

\hformbar



\formdesc{Polynomial Functions}

\begin{equation}
  f(x) = a_0 + a_1x + a_2x^2 + \cdots + a_nx^n = \sum_{k=0}^n a_k x^{n-k}
\end{equation}

\begin{center}
  \begin{tabular}{ll}
    $a_0$     & Constant term                         \\
    $a_n$     & Polynomial coefficient                \\
    $a_nx^m$  & Term ($m$ is called \textit{degree})  \\
  \end{tabular}
\end{center}

\hformbar



\formdesc{Rational Functions}

Can be written as a quotient of two polynomials $P(x)$ and $Q(x)$:

\begin{equation}
  f(x) = \frac{P(x)}{Q(x)} = \frac{\sum_{k=0}^n a_k x^{n-k}}{\sum_{k=0}^n b_k x^{n-k}}
\end{equation}

  \begin{itemize}
    \item Horizontal Intercept is the inputs where the output is 0
    \item Vertical Intercept is where input is 0 (if defined)        
  \end{itemize}

\hformbar



\formdesc{Asymptotes}

The ``line'' a function approaches but never touches

\begin{description}
	\item[Vertical] A vertical line $x = a$ where the graph tends towards positive or negative 
	                infinity as the inputs approach $a$:
	                \begin{equation}
	                    x \rightarrow a,\, f(x) \rightarrow \pm~ \infty
	                \end{equation}
	\item[Horizontal] A horizontal line $y = b$ where the graph approaches the line as the
	                  input gets larger:
	                   \begin{equation}
	                       x \rightarrow \pm~ \infty,\, f(x) \rightarrow b
	                   \end{equation}
\end{description}
\hformbar



\formdesc{Asymptotes of Rational Functions}

\begin{description}
    \item[Vertical] Where denominator = 0 but numerator $\neq 0$
    \item[Horizontal] Determined by respective degrees of numerator and denominator:
        \begin{itemize}
            \item Degree of denominator $>$ degree of numerator $\Rightarrow$ Horizontal asymptote at $y = 0$
            \item Degree of denominator $<$ degree of numerator $\Rightarrow$ No horizontal asymptote
            \item Degree of denominator = degree of numerator $\Rightarrow$ Horizontal asymptote is ratio of leading coefficients
        \end{itemize}
\end{description}
\hformbar



\formdesc{Exponential Functions}

Rate of change is as a percent, i.e., not an constant (absolute) rate. Takes the form:

\begin{equation}
    f(x) = a(1 + r)^x
\end{equation}

or, 

\begin{equation}
    f(x) = ab^x,~ b = 1 + r
\end{equation}

Can always be rewritten in terms of logarithms:

\begin{equation}
    b^a = c \equiv log_b~ c = a
\end{equation}

\begin{description}
    \item[Continuous Growth] Use $e = 2.718282\dots$ for continuous growth, often natural phenomena
        \begin{equation}
            f(x) = ae^{rx}
        \end{equation}
        
        where
        
        \begin{itemize}
            \item $a \equiv$ initial quantity
            \item $r \equiv$ continuous growth rate
        \end{itemize}
\end{description}

\textit{ALMOST ALWAYS USE CONTINUOUS $e$ FORM!}\\

Solve by:

\begin{enumerate}
    \item Isolate exponential expression, where possible
    \item Take log on both sides
    \item Use \textit{exponent property} of logs to pull variables of out exponent
    \item Use algebra to solve for variable
\end{enumerate}
\hformbar



\formdesc{Graphing Exponential Functions}

\begin{itemize}
    \item $a \equiv$ vertical intercept
        \begin{itemize}
            \item $a > 0 \Rightarrow$ concave up
            \item $a < 0 \Rightarrow$ concave down
        \end{itemize}
    \item $b \equiv$ rate of growth
        \begin{itemize}
            \item $b > 1 \Rightarrow$ growing
            \item $0 < b < 1 \Rightarrow$ decaying
        \end{itemize}
    \item Horizontal asymptote is where $y = 0$
\end{itemize}
\hformbar



\formdesc{Logarithmic Functions}

The inverse of exponential functions; use to solve exponential functions. Commonly used to express quantities that vary widely in size. Form:

\begin{equation}
    log_b~ x
\end{equation}

which can be rewritten in terms of exponents:

\begin{equation}
    b^a = c \equiv log_b~ c = a
\end{equation}

\begin{description}
    \item[Inverse Property of Logs] 
        \begin{eqnarray}
            log_b~ b^x &=& x \\
            b^{log_b~ x} &=& x
        \end{eqnarray}
        
    \item[Exponent Property]
        \begin{equation}
            log_b~ A^r = r log_b~ A
        \end{equation}
\end{description}



\newpage
